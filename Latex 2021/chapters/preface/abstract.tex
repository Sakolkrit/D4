%%WWWWWWWWWWWWWWWWWWWWWWWWWWWWWWWWWWWWWWWWWWWWWWWWWWWWWWWWWWWWWWWWWWWWWWWWW%%
%%%%%%%%%%%%%%%%%%%%%%%%%%%%%%%%%%%%%%%%%%%%%%%%%%%%%%%%%%%%%%
%%%%%%%%%%%%%%%%%%%%%%         English abstract         %%%%%%%%%%%%%%%%%%%%%%%
%%%%%%%%%%%%%%%%%%%%%%%%%%%%%%%%%%%%%%%%%%%%%%%%%%%%%%%%%%%%%%
%%WWWWWWWWWWWWWWWWWWWWWWWWWWWWWWWWWWWWWWWWWWWWWWWWWWWWWWWWWWWWWWWWWWWWWWWWW%%
\abstract

In a multihop ad hoc network, the interference among nodes is
  reduced to maximize the throughput by using a smallest transmission
  range that still preserve the network connectivity. However, most
  existing works on transmission range control focus on the
  connectivity but lack of results on the throughput performance. This
  paper analyzes the per-node saturated throughput of an IEEE 802.11b
  multihop ad hoc network with a uniform transmission range. Compared
  to simulation, our model can accurately predict the per-node
  throughput.  The results show that the maximum achievable per-node
  throughput can be as low as 11\% of the channel capacity in a normal
  set of $\alpha$ operating parameters independent of node density. However, if
  the network connectivity is considered, the obtainable throughput
  will reduce by as many as 43\% of the maximum throughput. YOOO

\begin{flushleft}
\begin{tabular*}{\textwidth}{@{}lp{0.8\textwidth}}
\textbf{Keywords}: & Multihop ad hoc networks / Topology control / Single-Hop Throughput
\end{tabular*}
\end{flushleft}
\endabstract